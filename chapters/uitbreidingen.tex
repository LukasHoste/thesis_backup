\chapter{Uitbreidingen}\label{ch:uitbreidingen}

De webapp bevat alle belangrijkste functionaliteiten. We zouden nog enkele concepten, zoals de ''melanopic eye sensitivity'', kunnen toevoegen. Dit concept meet hoe gevoelig het menselijk oog is voor licht. Toch hebben we besloten om dit en andere complexe concepten niet op te nemen, omdat ze de webapp onnodig ingewikkeld zouden maken. Ons doel is om de webapp zo eenvoudig en toegankelijk mogelijk te houden.

Naast het toevoegen van concepten kunnen we de webapp ook uitbreiden met extra functionaliteiten. Zo kunnen we bijvoorbeeld gebruikers de mogelijkheid geven om een account aan te maken. Met dit account kunnen zij instellingen, \gls{led}-configuraties en andere persoonlijke voorkeuren permanent opslaan.

Daarnaast kunnen we het educatieve aspect van de webapp verder uitbreiden. Momenteel ligt de focus vooral op interactiviteit en experimentatie. We kunnen echter extra educatieve elementen toevoegen, zoals uitleg over de gebruikte concepten en oefeningen om de leerervaring te verdiepen.

\section{Gamification}

Zoals eerder vermeld, kan de webapp uitgebreid worden met meer educatieve aspecten. Gamification is een populaire methode om dit te realiseren ~\cite{saleemGamificationApplicationsElearning2022}. Dit houdt in dat we game-elementen toevoegen aan een educatieve app, zonder dat het een volledige game wordt. In tegenstelling tot ''Game-based learning'', waarbij leren plaatsvindt door een spel te spelen, verhoogt gamification de motivatie en aandacht van de gebruiker door spelelementen te integreren. Hierdoor leert de gebruiker meer en beter.

Veelgebruikte game-elementen zijn:
\begin{itemize}
  \item punten
  \item uitdagingen
  \item levels
  \item klassementen
  \item badges
\end{itemize}

Deze elementen houden niet alleen de gebruiker gemotiveerd, maar stimuleren ook interactie en discussie tussen gebruikers of studenten. Dit bevordert wederzijds leren en kan leiden tot een gezonde competitie binnen de gebruikersgroep.

\section{Analyse model luminescente materialen}

Het geavanceerde model (\cref{sec:advanced}) voor de luminescente materialen is momenteel ge\"implementeerd in de app, maar nog niet vergeleken met experimentele data. De app gebruikt dit model uitsluitend op basis van het beschikbare bewijs. Daarom is het belangrijk om het model te toetsen aan experimentele data. Dit bevestigt de correctheid en biedt de mogelijkheid om het model verder te optimaliseren.

Voor de excitatie- en emissiespectra gebruiken we eenvoudige formules, maar deze kunnen we uitbreiden met geavanceerdere formules die beter aansluiten op de realiteit. Daarnaast kunnen we meerdere standaardformules aanbieden, zodat zowel algemene als specifieke formules beschikbaar zijn. De gebruiker kiest vervolgens zelf welke formule hij of zij wil gebruiken.




